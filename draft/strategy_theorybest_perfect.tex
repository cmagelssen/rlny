We found no statistically significant differences between reinforcement learning and supervised (free choice) learning in free choice 1 (0.01, 95\%CI[-0.24, 0.27], $z$ $=$ 0.11, $p$ $=$ 0.911), the first acquisition session where athletes or coaches were given autonomy to choose strategy. Moreover, we did not find a significantly increased probability of selecting this strategy in either reinforcement learning or supervised (free-choice) learning for free-choice 2, nor was the increase in descriptive greater change in supervised (learning) learning significant. However, we found this during retention, when athletes in supervised (free choice) learning were given autonomy to choose strategy themselves. Here, we found that supervised (free choice) learning significantly increased their selection of the theoretically best strategy with reference to free choice 1, which was a statistically significant greater increase than that of the reinforcement learning group, which also did not significantly increase during this session. However, reinforcement learning followed the transfer test, with a statistically significant increase from free choice 1. Despite large descriptive differences between reinforcement learning and supervised (free choice) learning in retention and transfer, none of the differences between the groups at each session were statistically significant.