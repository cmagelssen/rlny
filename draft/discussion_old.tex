a standard coaching practice where a coach chose a strategy for the skier.





To address this question, we delineated four strategies grounded in mechanics aiming to improve race time on flat sections in slalom and allocated ninety-eight skilled alpine ski racers to one of three treatment groups who learned the strategy choices in different ways. The skiers in the reinforcement learning group were tasked with finding the best of these strategies themselves using trial feedback for evaluation, while those in the supervised (free choice) learning group were paired with a coach from the tested ski teams who selected strategies for the skier and offered feedback on execution. Finally, in the supervised (target skill) learning group, the skiers were assigned a current national team coach who coached them in the 'extend with rock skis forward' skiing strategy — a technique we defined as the theoretically best strategy based on observations of elite skiers, simulations and theory. Overall, we found that skiers in the reinforcement learning group improved more during the acquisition sessions and performed better during retention than the supervised (free choice) learning group. We did not find evidence that t





to coach in the 'extend with rock skis forward' skiing strategy — a technique we defined as the theoretically best strategy based on observations of elite skiers, simulations and theory. 


with the potential to improve race time on flat sections in slalom and allocated ninety-eight skilled alpine ski racers to one of three treatment groups who learned the strategy choices in different ways.



The skiers in the reinforcement learning group were tasked with finding the best of these strategies themselves using trial feedback for evaluation. The skiers in the supervised (free choice) learning group were assigned to a coach from the tested ski teams who selected a strategy for the skier and provided feedback on their execution. Finally, in the supervised (target skill) learning group, the skiers were assigned a current national team coach to coach in the 'extend with rock skis forward' skiing strategy — a technique we defined as the theoretically best strategy based on observations of elite skiers, simulations and theory. Overall, we found that reinforcement learning learned better than the supervised (free choice) learning group [usikker på om jeg skal ta med noe mer hva vi fant her faktisk?]




Our work sought to explain these differences in race times through the choice of strategies. We did not, however,  find convincing evidence that the reinforcement learning group selected the theoretical best or the individual skiers' estimated best strategy more often than the supervised (free choice) learning group. The choices were also characterized by a clear 'win-stay, lose-switch' signature in both groups, where the chooser opted to stick with choices that yielded good outcomes. However, despite significant descriptive differences favoring reinforcement learning, this pattern was not statistically significant either. An obvious explanation for this was the high coaching experience the coaches in the study possessed, as well as their access to the outcomes and undergoing considerable learning themselves. 


The vigor perspective may also help explaining why reinforcement learning did not learn better than the supervised (target skill) learning group, as previous studies also have found that training with explicit knowledge boosts motor vigor much like the effect of reward itself \cite{anderson_rewards_2020, wong_explicit_2015}. It may therefore be that the getting information from a current national team coach that one strategy was best boosted the implicit motivation to perform this strategy well. 


Et overraskende funn var at, selv om det var bred enighet om at 'extend with rock skis forward' var den beste strategien, var det stort overlapp i hva skikjørerende og og trenerne mente var den viktigste underliggende mekanismen til hvorfor. 


A surprising and interesting discovery was that supervised (target skill) learning, in contrast to the reinforcement learning group, did not learn to cognitively dissociate the 'extend' and 'rock skis forward' strategies, despite big race time differences. This finding suggests that learners may miss potential learning by only being exposed to one strategy instead of a broader exploration of alternatives. Over time, such exploration could prove crucial in developing innovative strategies, as athletes cultivate a deeper comprehension of the relationship between their actions and performance outcomes \cite{ericsson_scientific_1998}. Future studies should investigate this learning further. 


One part of the explanation for why the reinforcement learning group learned more than the supervised (free choice) learning group could also be that they improved more on a strategy they chose frequently.  We found that the reinforcement learning group improved more on the 'extend' strategy over the course of the sessions than the supervised (free choice) learning group. Given that this group was not assigned a coach to help them improve, it is likely that the reinforcement feedback increased motor vigor \cite{shadmehr_vigor_2020, pietro_mazzoni_why_2007, niv_normative_2006} when performing that strategy. Indeed, previous studies have shown that people make saccades \cite{takikawa_modulation_2002} and reach faster \cite{summerside_vigor_2018} toward targets paired with rewards than unpaired targets. We found that there was a low proportion of athletes in the reinforcement learning group who performed best with 'extend with rock skis forward' compared to the supervised (free choice) learning group. It may therefore appear that more skiers opted for 'extend' instead of 'extend with rock skis forward' but executed it with greater force and effort. Comments from a few coaches, who watched the retention and transfer from the sideline, mentioned that skiers in the reinforcement learning group used more forceful arm movements than skiers in the other groups, although the instructions did not explicitly tell them to do that.