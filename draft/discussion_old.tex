a standard coaching practice where a coach chose a strategy for the skier.





To address this question, we delineated four strategies grounded in mechanics aiming to improve race time on flat sections in slalom and allocated ninety-eight skilled alpine ski racers to one of three treatment groups who learned the strategy choices in different ways. The skiers in the reinforcement learning group were tasked with finding the best of these strategies themselves using trial feedback for evaluation, while those in the supervised (free choice) learning group were paired with a coach from the tested ski teams who selected strategies for the skier and offered feedback on execution. Finally, in the supervised (target skill) learning group, the skiers were assigned a current national team coach who coached them in the 'extend with rock skis forward' skiing strategy — a technique we defined as the theoretically best strategy based on observations of elite skiers, simulations and theory. Overall, we found that skiers in the reinforcement learning group improved more during the acquisition sessions and performed better during retention than the supervised (free choice) learning group. We did not find evidence that t





to coach in the 'extend with rock skis forward' skiing strategy — a technique we defined as the theoretically best strategy based on observations of elite skiers, simulations and theory. 


with the potential to improve race time on flat sections in slalom and allocated ninety-eight skilled alpine ski racers to one of three treatment groups who learned the strategy choices in different ways.



The skiers in the reinforcement learning group were tasked with finding the best of these strategies themselves using trial feedback for evaluation. The skiers in the supervised (free choice) learning group were assigned to a coach from the tested ski teams who selected a strategy for the skier and provided feedback on their execution. Finally, in the supervised (target skill) learning group, the skiers were assigned a current national team coach to coach in the 'extend with rock skis forward' skiing strategy — a technique we defined as the theoretically best strategy based on observations of elite skiers, simulations and theory. Overall, we found that reinforcement learning learned better than the supervised (free choice) learning group [usikker på om jeg skal ta med noe mer hva vi fant her faktisk?]




Our work sought to explain these differences in race times through the choice of strategies. We did not, however,  find convincing evidence that the reinforcement learning group selected the theoretical best or the individual skiers' estimated best strategy more often than the supervised (free choice) learning group. The choices were also characterized by a clear 'win-stay, lose-switch' signature in both groups, where the chooser opted to stick with choices that yielded good outcomes. However, despite significant descriptive differences favoring reinforcement learning, this pattern was not statistically significant either. An obvious explanation for this was the high coaching experience the coaches in the study possessed, as well as their access to the outcomes and undergoing considerable learning themselves. 