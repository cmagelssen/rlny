To address this question, we conducted a three-day learning experiment with ninety-eight skilled and elite alpine ski racers from Norway and Sweden. To achieve performance improvement among this skilled cohort of athletes, we chose to focus on flat sections in slalom, an area with considerable potential for enhancement even among the best performers\cite{supej_new_2011}. To further develop the skill of proficiently skiing flat sections, we delineated four strategies (Fig. \ref{fig:courseandstrategies}b), each carefully selected to enhance performance by being rooted in mechanics. Our hypothesis was that skiers in the reinforcement learning group would learn to choose better strategies and thus achieve better performance than skiers subject to traditional supervised learning with a coach. To test this, we assigned skiers to three different learning groups with different instructions and feedback (Fig. \ref{fig:experiment}b): In the reinforcement learning group, skiers chose a strategy on every run and saw their race times to inform these decisions. In the supervised (free choice) learning group, we recruited ski coaches from the tested ski teams to coach on the strategy they believed to be the best or most appropriate for the skier. In the supervised (target skill) learning, we recruited ski coaches to instruct skiers to select the strategy that we defined as the theoretically best strategy based on computational modelling \cite{lind_physics_2013, mote_accelerations_1983, luginbuhl_identification_2023} and observations of elite skiers \cite{reid_alpine_2020, magelssen_is_2022}. This group served as a benchmark for the upper limit of performance achievable through optimal strategy choices, and we did not expect to find differences in favor of reinforcement learning. The coaches in both supervised learning groups were highly experienced (Table \ref{descriptive_coach}). Coaches in the two supervised learning groups saw the times but were instructed to not disclose these to the skiers. 


We found that the reinforcement learning group improved more during acquisition and performed better in retention than the supervised (free choice) learning group. Both groups chose the individual skiers' estimated best strategy more often over the course of the sessions, but we did not find convincing evidence that the reinforcement learning group chose this strategy more often than the supervised (free choice) learning group. That said, we found that reinforcement learning had lower costs for suboptimally chosen strategies (that is, the expected difference between the individual skiers' estimated strategy and their chosen strategy), suggesting that they had better learned to avoid bad strategies. This was not the sole explanation for their improved learning, however. The skiers also improved more on one strategy that they picked often than the supervised (free choice) learning, suggesting that reinforcement learning also increased motor vigor. We did not find convincing evidence that reinforcement learning learned better than supervised learning (target skill), which was expected. However, selecting only one strategy all the time came at its own cost—the skiers in the supervised (target skill) learning did not learn to dissociate the effect of the other strategies despite large differences in race times.  






Reinforcement learning has been tremendously powerful in explaining human and animal learning \cite{waelti_dopamine_2001, schultz_neural_1997, pessiglione_dopamine-dependent_2006,lee_neural_2012, law_reinforcement_2009}, improving skill learning in laboratory-based tasks \cite{lior_shmuelof_overcoming_2012, abe_reward_2011, truong_error-based_2023, hasson_reinforcement_2015}, as well as training AI to perform complex tasks such as computer games starting from pixel inputs, only\cite{mnih_human-level_2015}. Based on this evidence, our question was whether reinforcement learning offers a better alternative for training learners to make better decisions about strategies than traditional supervised learning with a coach. 




Although supervised learning can be effective for training performers to make good decisions, the teaching strategy has its limitations. 

Selv om supervised learning kan være effektiv for å trene utøvere til å ta gode valg har metodene sine begrensninger.

One drawback of the supervised learning strategy for training these decisions is that the coach's advice may not be the optimal solution, as what coaches judge as a good strategy does not always align with reality, even for the best-trained eye \cite{supej_impact_2019, cochrum_visual_2021}. Performers might therefore miss opportunities to discover the best strategy when coaches opt for suboptimal strategies \cite{gray_plateaus_2017}. Worse, these learned suboptimal strategies might turn into habits that can be difficult to break \cite{popp_effect_2020}. Supervised learning might also constrain learners to adopting a single ('universal') strategy for all situations rather than acquiring a repertoire of strategies and discerning the most effective strategies for each specific scenario. Finally, it remains uncertain whether the prescriptive approach is the most effective teaching strategy for achieving long-lasting learning effects \cite{wulf_instructions_1997, hodges_role_1999, williams_practice_2005,williams_effective_2023}.
